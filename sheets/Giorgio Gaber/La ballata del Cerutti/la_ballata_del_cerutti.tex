\documentclass[10pt, twoside, a4paper]{article}
\usepackage[italian]{babel}
%\usepackage[utf8]{inputenc}
\usepackage{fullpage}
\usepackage{graphicx}
%\usepackage{booktabs}
%\usepackage{wrapfig}
%\usepackage{sidecap}
%\usepackage[small]{caption}
\usepackage{gchords}			%serve per scrivere gli accordi
\setlength{\parindent}{0in}	%serve per togliere l'indentazione della prima riga
\usepackage[usenames,dvipsnames,svgnames,table]{xcolor}	%serve per utilizzare i colori \textcolor{blue}{}
\usepackage[final]{pdfpages} %serve per aggiungere altre pagine pdf al file
\usepackage{multicol}

\begin{document}

\begin{center}

	\hrule \vspace{0.2cm}
     	\textsc{\LARGE La Ballata Del Cerutti}
	\vspace{0.2cm} \hrule \vspace{0.2cm}
      	{\large Giorgio Gaber}
      	
\end{center}

\chords{
		\chord{t}{x,o,o,p2,p3,p2}{D}
		\chord{t}{o,o,p2,o,p2,o}{A7}
		\chord{t}{o,p2,o,p1,o,o}{E7}
		\chord{t}{p3,p2,o,o,o,p3}{G}
		\chord{t}{o,p2,o,o,o,o}{Em7}
		}
		
\textbf{Rhythm:} $\uparrow \,\downarrow \,\downarrow$ mute \\

Io ho sentito molte ballate:\\
quella di Tom Dooley, quella di Davy Crockett...\\
e sarebbe piaciuto anche a me scriverne una cos\`i.\\
E invece, invece niente.\\
Ho fatto una ballata per uno\\
che sta a Milano, al Giambellino:\\
il Cerutti, il Cerutti Gino.\\

\begin{multicols}{2}

\hspace{8pt} \upchord{D}Il suo nome er\upchord{A7}a Cerutti \upchord{D}Gino

\hspace{8pt} ma lo chia\upchord{E7}mavan \upchord{A7}Drago.

\hspace{8pt} Gli a\upchord{D}mici, al \upchord{G}bar del Giambe\upchord{D}llino

\hspace{8pt} di\upchord{Em7}cevan ch\upchord{A7}'era un \upchord{D}mago (\upchord{G}era un m\upchord{D}ago)\hspace{8pt}\upchord{A7}\\

Vent'\upchord{D}anni, biondo, mai una lira,\\
per \upchord{E7}non passare \upchord{A7}guai\\
fiut\upchord{D}ava in\upchord{G}torno che aria \upchord{D}tira\\
e \upchord{Em7}non sgob\upchord{A7}bava \upchord{D}mai.\\

\hspace{8pt} Il suo nome era Cerutti Gino

\hspace{8pt} ma lo chiamavan Drago.

\hspace{8pt} Gli amici, al bar del Giambellino

\hspace{8pt} dicevan ch'era un mago (era un mago)\\

Una sera in una strada scura\\
occhio, c'è una Lambretta\\
Fingendo di non aver paura\\
il Cerutti monta in fretta.\\

SOL\hspace{10pt} RE\hspace{10pt} SOL\hspace{10pt} RE\hspace{10pt} LA7\\

Ma che rogna nera quella sera:\\
qualcuno vede e chiama;\\
veloce arriva la Pantera\\
e lo beve la Madama.\\
\columnbreak

\hspace{8pt} Il suo nome era Cerutti Gino

\hspace{8pt} ma lo chiamavan Drago.

\hspace{8pt} Gli amici, al bar del Giambellino

\hspace{8pt} dicevan ch'era un mago (era un mago)\\

Ora è triste e un poco manomesso:\\
si trova al terzo raggio.\\
E' lì che attende il suo processo,\\
forse vien fuori a maggio.\\

S'è beccato un bel tre mesi il Gino,\\
ma il giudice è stato buono:\\
gli ha fatto un lungo fervorino,\\
è uscito col condono.\\

\hspace{8pt} Il suo nome era Cerutti Gino

\hspace{8pt} ma lo chiamavan Drago.

\hspace{8pt} Gli amici, al bar del Giambellino

\hspace{8pt} dicevan ch'era un mago (era un mago)\\

E' tornato al bar Cerutti Gino\\
e gli amici nel futuro\\
quando parleran del Gino\\
diran che è un tipo duro!\\
\end{multicols}
\end{document}