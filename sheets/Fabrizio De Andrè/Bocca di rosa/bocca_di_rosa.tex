\documentclass[10pt, twoside, a4paper]{article}
\usepackage[italian]{babel}
%\usepackage[utf8]{inputenc}
\usepackage{fullpage}
\usepackage{graphicx}
%\usepackage{booktabs}
%\usepackage{wrapfig}
%\usepackage{sidecap}
%\usepackage[small]{caption}
\usepackage{gchords}			%serve per scrivere gli accordi
\setlength{\parindent}{0in}	%serve per togliere l'indentazione della prima riga
\usepackage[usenames,dvipsnames,svgnames,table]{xcolor}	%serve per utilizzare i colori \textcolor{blue}{}
\usepackage[final]{pdfpages} %serve per aggiungere altre pagine pdf al file
\usepackage{multicol}

\begin{document}

\begin{center}

	\hrule \vspace{0.2cm}
     	\textsc{\LARGE Bocca Di Rosa}
	\vspace{0.2cm} \hrule \vspace{0.2cm}
      	{\large Fabrizio de Andr\'e}
      	
\end{center}

%	{\begin{abstract}
%	blablalba
%	 \end{abstract}}

\chords{\chord{t}{x,o,p2,p2,p1,o}{Am}
		\chord{t}{o,p2,p2,p1,o,o}{E}
		\chord{t}{x,o,p2,o,p2,o}{A7}
		\chord{t}{x,o,o,p2,p3,p1}{Dm}
		\chord{{3~}}{p1,p3,p1,p2,p1,p1}{G7}
		\chord{t}{x,p3,p2,o,p1,o}{C}
		\chord{t}{o,p2,o,p1,o,o}{E7}

		}
		
\textbf{Rhythm:} %$\texttt{IV} \downarrow \texttt{IV} \downarrow \,\, \mid \texttt{V} \downarrow \texttt{V} \downarrow \, \uparrow \, \downarrow$ \\
\\

\begin{multicols}{2}

La chia\upchord{Am}mavano bocca di \upchord{Am}rosa \\
metteva l'a\upchord{E}more, metteva l'a\upchord{Am}more, \\
la chia\upchord{Am}mavano bocca di \upchord{Am}rosa \\
metteva l'a\upchord{E}more sopra ogni \upchord{Am}cosa. \\

Appena scese alla stazione \\
nel paesino di Sant'Ilario \\
tutti si accorsero con uno sguardo \\
che non si trattava di un missionario. \\

C'\`e chi l'a\upchord{A7}more lo fa per \upchord{Dm}noia \\
chi se lo \upchord{G7}sceglie per professi\upchord{C}one \\
bocca di \upchord{Dm}rosa n\'e l'uno n\'e l'a\upchord{Am}ltro \\
lei lo fa\upchord{E7}ceva per passi\upchord{Am}one. \\

Ma la passione spesso conduce \\
a soddisfare le proprie voglie \\
senza indagare se il concupito \\
ha il cuore libero oppure ha moglie. \\

E fu cos\`i che da un giorno all'altro \\
bocca di rosa si tir\`o addosso \\
l'ira funesta delle cagnette \\
a cui aveva sottratto l'osso. \\

Ma le co\upchord{A7}mari di un pae\upchord{Dm}sino \\
non brillano \upchord{G7}certo inizia\upchord{C}tiva \\
le contromi\upchord{Dm}sure fino a quel \upchord{Am}punto \\
si limi\upchord{E7}tavano all'invet\upchord{Am}tiva. \\

Lam Rem/ Sol/ Do/ Lam Rem/ Mi/ Lam/
\columnbreak

Si sa che la gente d\`a buoni consigli \\
sentendosi come Ges\`u nel tempio, \\
si sa che la gente d\`a buoni consigli \\
se non pu\`o pi\`u dare cattivo esempio. \\

Cos\`i una vecchia mai stata moglie \\
senza mai figli, senza pi\`u voglie, \\
si prese la briga e di certo il gusto \\
di dare a tutte il consiglio giusto. \\

E rivol\upchord{A7}gendosi alle cor\upchord{Dm}nute \\
le apostro\upchord{G7}f\`o con parole ar\upchord{C}gute: \\
"il furto d'a\upchord{Dm}more sar\`a pun\upchord{Am}ito- \\
disse- dall'\upchord{E7}ordine costitu\upchord{Am}ito". \\

E quelle andarono dal commissario \\
e dissero senza parafrasare: \\
"quella schifosa ha gi\`a troppi clienti \\
pi\`u di un consorzio alimentare". \\

E arrivarono quattro gendarmi \\
con i pennacchi con i pennacchi \\
e arrivarono quattro gendarmi \\
con i pennacchi e con le armi. \\

spesso gli s\upchord{A7}birri e i carabin\upchord{Dm}ieri \\
al proprio do\upchord{G7}vere vengono m\upchord{C}eno, \\
ma non quando \upchord{Dm}sono in alta unif\upchord{Am}orme \\
e l'accompag\upchord{E7}narono al primo tr\upchord{Am}eno. \\

\textit{Variazione}:

\hspace{36pt}Il cuore tenero non \`e una dote

\hspace{36pt}di cui sian colmi i carabinieri

\hspace{36pt}ma quella volta a prendere il treno

\hspace{36pt}l'accompagnarono malvolentieri.\\

Lam Rem/ Sol/ Do/ Lam Rem/ Mi/ Lam/
\end{multicols}
\newpage


Alla stazione c'erano tutti \\
dal commissario al sagrestano \\
alla stazione c'erano tutti \\
con gli occhi rossi e il cappello in mano, \\

a salutare chi per un poco \\
senza pretese, senza pretese, \\
a salutare chi per un poco \\
port\`o l'amore nel paese. \\

C'era un\upchord{A7} cartello \upchord{Dm}giallo \\
con una scr\upchord{G7}itta nera\upchord{C} \\
diceva "Add\upchord{Dm}io bocca di \upchord{Am}rosa \\
con te se ne \upchord{E7}parte la prima\upchord{Am}vera". \\

Ma una notizia un po' originale \\
non ha bisogno di alcun giornale \\
come una freccia dall'arco scocca \\
vola veloce di bocca in bocca. \\

E alla stazione successiva \\
molta pi\`u gente di quando partiva \\
chi mand\`o un bacio, chi gett\`o un fiore \\
chi si prenota per due ore. \\

Persino il \upchord{A7}parroco che non dispr\upchord{Dm}ezza \\
fra un mis\upchord{G7}erere e un'estrema unz\upchord{C}ione \\
il bene eff\upchord{Dm}imero della bell\upchord{Am}ezza \\
la vuole acc\upchord{E7}anto in process\upchord{Am}ione. \\

E con la Vergine in prima fila \\
e bocca di rosa poco lontano \\
si porta a spasso per il paese \\
l'amore sacro e l'amor profano. \\

Lam Rem/ Sol/ Do/ Lam Rem/ Mi/ Lam/

\end{document}