\documentclass[10pt, twoside, a4paper]{article}
\usepackage[italian]{babel}
%\usepackage[utf8]{inputenc}
\usepackage{fullpage}
\usepackage{graphicx}
%\usepackage{booktabs}
%\usepackage{wrapfig}
%\usepackage{sidecap}
%\usepackage[small]{caption}
\usepackage{gchords}			%serve per scrivere gli accordi
\setlength{\parindent}{0in}	%serve per togliere l'indentazione della prima riga
\usepackage[usenames,dvipsnames,svgnames,table]{xcolor}	%serve per utilizzare i colori \textcolor{blue}{}
\usepackage[final]{pdfpages} %serve per aggiungere altre pagine pdf al file
\usepackage{multicol}

\begin{document}

\begin{center}

	\hrule \vspace{0.2cm}
     	\textsc{\LARGE La Locomotiva}
	\vspace{0.2cm} \hrule \vspace{0.2cm}
      	{\large Francesco Guccini}
      	
\end{center}

%	{\begin{abstract}
%	blablalba
%	 \end{abstract}}

\chords{\chord{t}{x,o,o,p2,p3,p2}{D}
		\chord{t}{x,o,p2,o,p2,o}{A7}
		\chord{t}{p3,p2,o,o,o,p3}{G}
		\chord{t}{x,o,p3,p3,p3,o}{A}
		\chord{t}{p2,p4,p4,p2,p2,p2}{F\#m}
		\chord{t}{x,p2,p4,p4,p3,p2}{Bm}
		}
\chords{
		\chord{t}{o,p2,p2,p1,o,o}{E}
		\chord{t}{x,p2,p4,p2,p4,p2}{B7}
		\chord{t}{x,p2,p4,p4,p4,p2}{B}
		\chord{4~}{p1,p3,p3,p1,p1,p1}{G\#m}
		\chord{4~}{x,p1,p3,p3,p2,p1}{C\#m}		
		}
\chords{
		\chord{t}{p1,p3,p3,p2,p1,p1}{F}
		\chord{3~}{x,p1,p3,p1,p3,p1}{C7}
		\chord{t}{p1,p1,p3,p3,p3,p1}{Bb}
		\chord{3~}{x,p1,p3,p3,p3,p1}{C}
		\chord{t}{x,o,p2,p2,p1,o}{Am}
		\chord{t}{x,o,o,p2,p3,p1}{Dm}
		}

		
\textbf{Rhythm:} %$\texttt{IV} \downarrow \texttt{IV} \downarrow \,\, \mid \texttt{V} \downarrow \texttt{V} \downarrow \, \uparrow \, \downarrow$ \\
\\

%\begin{multicols}{2}
%\textit{Variazione}:\\
%\textcolor{white}{}\hspace{36pt}l'accompagnarono malvolentieri. \\
%\end{multicols}

\textbf{Intro:} DO SOL7 DO SOL7\\

Non \upchord{D}so che viso avesse, nep\upchord{A7}pure come si chi\upchord{D}amava,  \upchord{A7}\\
con \upchord{D}che voce parlasse, con \upchord{A7}quale voce poi ca\upchord{D}ntava, \upchord{A7}\\
quanti \upchord{G}anni avesse v\upchord{A}isto al\upchord{D}lora, di \upchord{G}che col\upchord{A}ore i suoi ca\upchord{D}pelli,\\
ma \upchord{G}nella fantas\upchord{A}ia ho l'im\upchord{FA\#m}magine sua\upchord{Bm}:\\
gli \upchord{G}eroi son tutti \upchord{A}giovani e \upchord{D}belli,\\
gli \upchord{G}eroi son tutti \upchord{A}giovani e \upchord{D}belli,\\
gli \upchord{G}eroi son tutti \upchord{A}giovani e \upchord{D}belli...\\

Conosco invece l'epoca dei fatti, qual' era il suo mestiere:\\
i primi anni del secolo, macchinista, ferroviere,\\
i tempi in cui si cominciava la guerra santa dei pezzenti\\
sembrava il treno anch' esso un mito di progresso\\
lanciato sopra i continenti,\\
lanciato sopra i continenti,\\
lanciato sopra i continenti...\\

E la locomotiva sembrava fosse un mostro strano\\
che l'uomo dominava con il pensiero e con la mano:\\
ruggendo si lasciava indietro distanze che sembravano infinite,\\
sembrava avesse dentro un potere tremendo,\\
la stessa forza della dinamite,\\
la stessa forza della dinamite,\\
la stessa forza della dinamite...\\

Ma un' altra grande forza spiegava allora le sue ali,\\
parole che dicevano "gli uomini son tutti uguali"\\
e contro ai re e ai tiranni scoppiava nella via\\
la bomba proletaria e illuminava l' aria\\
la fiaccola dell' anarchia,\\
la fiaccola dell' anarchia,\\
la fiaccola dell' anarchia...\\

Un treno tutti i giorni passava per la sua stazione,\\
un treno di lusso, lontana destinazione:\\
vedeva gente riverita, pensava a quei velluti, agli ori,\\
pensava al magro giorno della sua gente attorno,\\
pensava un treno pieno di signori,\\
pensava un treno pieno di signori,\\
pensava un treno pieno di signori...\\

Non so che cosa accadde, perchè prese la decisione,\\
forse una rabbia antica, generazioni senza nome\\
che urlarono vendetta, gli accecarono il cuore:\\
dimenticò pietà, scordò la sua bontà,\\
la bomba sua la macchina a vapore,\\
la bomba sua la macchina a vapore,\\
la bomba sua la macchina a vapore...\\

\textbf{Tonality change: E}\\

\upchord{E}E sul binario \upchord{B7}stava la locomo\upchord{E}tiva,   \upchord{B7}\\
la \upchord{E}macchina pulsante sem\upchord{B7}brava fosse cosa \upchord{E}viva,   \upchord{B7}\\
sem\upchord{A}brava un giovan\upchord{B}e pu\upchord{E}ledro che \upchord{A}appena libe\upchord{B}rato il \upchord{E}freno\\
mor\upchord{A}desse la ro\upchord{B}taia con \upchord{G\#m}muscoli d' ac\upchord{C\#m}ciaio,\\
con \upchord{A}forza cie\upchord{B}ca di ba\upchord{E}leno,\\
con \upchord{A}forza cie\upchord{B}ca di ba\upchord{E}leno,\\
con \upchord{A}forza cie\upchord{B}ca di ba\upchord{E}leno...\\

E un giorno come gli altri, ma forse con più rabbia in corpo\\
pensò che aveva il modo di riparare a qualche torto.\\
Salì sul mostro che dormiva, cercò di mandar via la sua paura\\
e prima di pensare a quel che stava a fare,\\
il mostro divorava la pianura,\\
il mostro divorava la pianura,\\
il mostro divorava la pianura...\\

Correva l' altro treno ignaro e quasi senza fretta,\\
nessuno immaginava di andare verso la vendetta,\\
ma alla stazione di Bologna arrivò la notizia in un baleno:\\
"notizia di emergenza, agite con urgenza,\\
un pazzo si è lanciato contro al treno,\\
un pazzo si è lanciato contro al treno,\\
un pazzo si è lanciato contro al treno..."\\

\textbf{Tonality change: F}\\

Ma in\upchord{F}tanto corre, corre, \upchord{C7}corre la locomo\upchord{F}tiva    \upchord{C7}\\
e \upchord{F}sibila il vapore e \upchord{C7}sembra quasi cosa \upchord{F}viva     \upchord{C7}\\
e \upchord{Bb}sembra dire ai \upchord{C}contadini \upchord{F}curvi il \upchord{Bb}fischio che \upchord{C}si spande in \upchord{F}aria:\\
"Fra\upchord{Bb}tello, non \upchord{C}temere, che \upchord{Am}corro al mio \upchord{Dm}dovere!\\
Tri\upchord{Bb}onfi la gius\upchord{C}tizia prole\upchord{F}taria!\\
Tri\upchord{Bb}onfi la gius\upchord{C}tizia prole\upchord{F}taria!\\
Tri\upchord{Bb}onfi la gius\upchord{C}tizia prole\upchord{F}taria!"\\

\textbf{Tonality change: E}\\
E intanto corre corre corre sempre più forte\\
e corre corre corre corre verso la morte\\
e niente ormai può trattenere l' immensa forza distruttrice,\\
aspetta sol lo schianto e poi che giunga il manto\\
della grande consolatrice,\\
della grande consolatrice,\\
della grande consolatrice...\\

\textbf{Tonality change: D}\\
La storia ci racconta come finì la corsa\\
la macchina deviata lungo una linea morta...\\
con l' ultimo suo grido d' animale la macchina eruttò lapilli e lava,\\
esplose contro il cielo, poi il fumo sparse il velo:\\
lo raccolsero che ancora respirava,\\
lo raccolsero che ancora respirava,\\
lo raccolsero che ancora respirava...\\

Ma a noi piace pensarlo ancora dietro al motore\\
mentre fa correr via la macchina a vapore\\
e che ci giunga un giorno ancora la notizia\\
di una locomotiva, come una cosa viva,\\
lanciata a bomba contro l' ingiustizia,\\
lanciata a bomba contro l' ingiustizia,\\
lanciata a bomba contro l' ingiustizia\\

\end{document}